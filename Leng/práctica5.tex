\documentclass[11pt]{article}
\usepackage[spanish]{babel}
\usepackage[utf8]{inputenc}
\usepackage{amsmath, amssymb}
\usepackage{geometry}
\usepackage{setspace}
\usepackage{enumitem}
\usepackage{graphicx}
\usepackage{xurl}
\usepackage{listings}
\usepackage{tikz}
\usepackage{xcolor}
\usetikzlibrary{decorations.pathreplacing, calc}

\geometry{letterpaper, margin=2cm}

\begin{document}
\begin{titlepage}
    \centering
    \vspace{2cm}
    {\includegraphics[height=3.2cm]{../logo_unam.png}}
    \hfill
    {\includegraphics[height=3.2cm]{../logo_fc.png}\par}
    \vspace{1cm}
    {\bfseries\LARGE UNIVERSIDAD NACIONAL AUTÓNOMA DE MÉXICO \par}
    \vspace{0.7cm}
    {\scshape\Large FACULTAD DE CIENCIAS \par}
    \vspace{1cm}
    {\itshape\Large Lenguajes de programación \par}
    \vspace{0.5cm}
    {\itshape\Large Semestre 2026-1 \par}
    \vspace{2cm}
    {\scshape\Huge Práctica 5 \par}
    \vspace{1cm}
    {\itshape\Large Fecha de entrega: 22 de octubre de 2025 \par}
    \vspace{2cm}
    {\Large Autores: \par}
    \vspace{0.4cm}
    {\Large Escobar Gonzalez Isaac Giovani \hspace{1cm} 321336400 \par}
    {\Large Garduño Escobar Kevin Jonathan \hspace{0.5cm} 321070629 \par}
    {\Large Zaldivar Alanis Rodrigo \hspace{2.75cm} 424029605 \par}
\end{titlepage}
\section{Objetivos}
Analizar las implementaciones dadas en los archivos \texttt{\textbf{grammars.rkt}}, \texttt{\textbf{parser.rkt}}, \texttt{\textbf{desugar.rkt}} e \texttt{\textbf{interp.rkt}} para el lenguaje \textbf{CFWBAE} y realizar los siguientes ejercicios.

\section{Ejercicios}
\subsection{Parser}
\begin{enumerate}
    \item (0.5 pts.) ¿Por qué es importante representar el programa como un árbol de sintaxis abstracta?
    \item (1 pts.) Explica cómo el parser reconoce una expresión \texttt{if} y la diferencia respecto al antiguo \texttt{if0}.
    \item (0.5 pts.) De acuerdo a la implementación en el archivo \texttt{ \textbf{parser.rkt}} ¿Qué ocurre si el parser recibe una expresión con paréntesis mal colocados o una forma desconocida?
    \subsection{Desugar}
    \item (1 pts.) ¿Qué significa “desazúcar” una expresión? Da un ejemplo.
    \item (1 pts.) ¿Cómo traduce cond una cascada de condiciones en una serie de if anidados? ¿Qué ocurre si no hay un else final?
    \item (1 pts.) ¿Qué pasaría si olvidas aplicar desugar recursivamente dentro de las subexpresiones (por ejemplo, en el cuerpo de un with)?
    \item (1 pts.) Si agregas un nuevo azúcar como \texttt{\textbf{when}}, ¿qué regla de traducción definirías en \texttt{\textbf{desugar.rkt}}?
    \item (1 pts.) ¿Por qué desugar siempre debe producir una expresión válida del lenguaje base \texttt{\textbf{(CFWBAE)}}?
    \subsection{Interp}
    \item (1 pts.) ¿Qué papel cumple el ambiente \texttt{\textbf{(DefrdSub)}} en la evaluación de expresiones con variables?
    \item (1 pts.) ¿Qué diferencias encuentras entre evaluar with directamente y evaluar la versión desazucarada \texttt{\textbf{(como app(fun ...))}}?
    \item (1 pts.) De acuerdo a la implementación dada en el archivo \texttt{\textbf{interp.rkt}} ¿Qué estrategia de evaluación usa el intérprete: estricta o perezosa? Justifica tu respuesta.
\end{enumerate}

\end{document}
