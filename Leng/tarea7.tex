\documentclass[11pt]{article}
\usepackage[spanish]{babel}
\usepackage[utf8]{inputenc}
\usepackage{amsmath, amssymb}
\usepackage{geometry}
\usepackage{setspace}
\usepackage{enumitem}
\usepackage{graphicx}
\usepackage{xurl}
\usepackage{listings}
\usepackage{tikz}
\usepackage{xcolor}
\usetikzlibrary{decorations.pathreplacing, calc}

\geometry{letterpaper, margin=2cm}

\begin{document}
\begin{titlepage}
    \centering
    \vspace{2cm}
    {\includegraphics[height=3.2cm]{../logo_unam.png}}
    \hfill
    {\includegraphics[height=3.2cm]{../logo_fc.png}\par}
    \vspace{1cm}
    {\bfseries\LARGE UNIVERSIDAD NACIONAL AUTÓNOMA DE MÉXICO \par}
    \vspace{0.7cm}
    {\scshape\Large FACULTAD DE CIENCIAS \par}
    \vspace{1cm}
    {\itshape\Large Lenguajes de programación \par}
    \vspace{0.5cm}
    {\itshape\Large Semestre 2026-1 \par}
    \vspace{2cm}
    {\scshape\Huge Tarea 7 \par}
    \vspace{1cm}
    {\itshape\Large Fecha de entrega: 20 de noviembre de 2025 \par}
    \vspace{2cm}
    {\Large Autores: \par}
    \vspace{0.4cm}
    {\Large Escobar Gonzalez Isaac Giovani \hspace{1cm} 321336400 \par}
    {\Large Garduño Escobar Kevin Jonathan \hspace{0.5cm} 321070629 \par}
    {\Large Zaldivar Alanis Rodrigo \hspace{2.75cm} 424029605 \par}
\end{titlepage}
\section*{Instrucciones}
\noindent Resolver los siguientes ejercicios de forma clara y ordenada de acuerdo a los lineamientos de entrega de tareas disponibles en la página del curso.\\
\section*{Ejercicios}

\begin{enumerate}[leftmargin=0.8cm]
    \item Observa la siguiente función
    \begin{lstlisting}
(define  (fun-x ls)
     (if (null? ls)
          0
          (+ (car ls
              (call/cc (lambda(k)
                   (if (< car ls) 0)
                        (k -100)
                        (fun-x (cdr ls))))))))
    \end{lstlisting}
    \begin{enumerate}
        \item ¿Qué hace la función \textit{fun-x}? Explica a profundidad la llamada de la continuación call/cc dentro del cuerpo de la función principal, ¿qué hace y para qué se requiere?
        \item ¿Qué regresa la función con la entrada \textbf{’(5 8 9)}? Justifica tu respuesta
        \item ¿Qué regresa la función con la entrada \textbf{’(-12 7 9)}? Justifica tu respuesta
    \end{enumerate}
    \item Observa la siguiente función
    \begin{lstlisting}
(define (funcion-y ls)
   (cond
      [(null? ls) 0]
      [else(+ (let ((n (car ls)))
             (let/cc k
                (cond
                   [(and (even? n) (> n 10)) (k 0)]
                   [(odd? n) n]
                   [else 0])))(funcion-y (cdr ls))])))
    \end{lstlisting}
    \begin{enumerate}
        \item ¿Qué hace la función \textit{fun-y}? Explica a profundidad la llamada de la continuación let/cc dentro del cuerpo de la función principal, ¿qué hace y para qué se requiere?
        \item ¿Qué regresa la función con la entrada \textbf{’(2 2 9)}? Justifica tu respuesta
        \item ¿Qué regresa la función con la entrada \textbf{’(-12 24 36)}? Justifica tu respuesta
    \end{enumerate}
    \item Evalúa el siguiente código en el \textit{lenguaje de programación Racket.} Explica su resultado y da la continuación asociada a evaluar, usando la notación $\lambda \uparrow$.
    \begin{lstlisting}
> (define c #f)

> (*1 (* 2 (* 3 (+ ( let / cc k ( set ! c k ) 4) 5))))

> (c 0)
    \end{lstlisting}
    \item Utiliza la técnica de \textit{Continuation Passing Style} para modificar la siguiente función.
    \begin{lstlisting}
(define (flatMap lst)
  (cond
    [(null? lst) '()]
    [else (append (car lst) (flatMap (cdr lst)))]))
    \end{lstlisting}
\end{enumerate}

\end{document}
