\documentclass[11pt]{article}
\usepackage[spanish]{babel}
\usepackage[utf8]{inputenc}
\usepackage{amsmath, amssymb}
\usepackage{geometry}
\usepackage{setspace}
\usepackage{enumitem}
\usepackage{graphicx}
\usepackage{xurl}
\usepackage{listings}
\usepackage{tikz}
\usepackage{xcolor}
\usepackage{longtable}
\usepackage{ mathrsfs }
\usepackage{bussproofs}
\usepackage{xcolor}
\usetikzlibrary{decorations.pathreplacing, calc}

\geometry{letterpaper, margin=2cm}

\lstset{escapeinside={(*}{*)}}

\begin{document}
\begin{titlepage}
    \centering
    \vspace{2cm}
    {\includegraphics[height=3.2cm]{../logo_unam.png}}
    \hfill
    {\includegraphics[height=3.2cm]{../logo_fc.png}\par}
    \vspace{1cm}
    {\bfseries\LARGE UNIVERSIDAD NACIONAL AUTÓNOMA DE MÉXICO \par}
    \vspace{0.7cm}
    {\scshape\Large FACULTAD DE CIENCIAS \par}
    \vspace{1cm}
    {\itshape\Large Lenguajes de programación \par}
    \vspace{0.5cm}
    {\itshape\Large Semestre 2026-1 \par}
    \vspace{2cm}
    {\scshape\Huge Tarea 8 \par}
    \vspace{1cm}
    {\itshape\Large Fecha de entrega: 01 de diciembre de 2025 \par}
    \vspace{2cm}
    {\Large Autores: \par}
    \vspace{0.4cm}
    {\Large Escobar Gonzalez Isaac Giovani \hspace{1cm} 321336400 \par}
    {\Large Garduño Escobar Kevin Jonathan \hspace{0.5cm} 321070629 \par}
    {\Large Zaldivar Alanis Rodrigo \hspace{2.75cm} 424029605 \par}
\end{titlepage}
\section*{Instrucciones}
\noindent Resolver los siguientes ejercicios de forma clara y ordenada de acuerdo a los lineamientos de entrega de tareas disponibles en la página del curso.\\
\section*{Ejercicios}

\begin{enumerate}[leftmargin=0.8cm]
    \item A partir de semántica operacional vista en clase realiza el árbol de derivación de la siguiente suma:\\
    $(1+2)+(2+(2+2))$\\
    Usando las reglas de reducción:
    \begin{center}
        \huge
        $
            \frac{\frac{1, \mathscr{E} \Rightarrow \hat{1} \quad 2, \mathscr{E} \Rightarrow \hat{2}}{(1+2), \mathscr{E} \Rightarrow \hat{1} + \hat{2} \Rightarrow \hat{3}} \qquad \frac{2, \mathscr{E} \Rightarrow \hat{2} \quad \frac{2, \mathscr{E} \Rightarrow \hat{2} \quad 2, \mathscr{E} \Rightarrow \hat{2}}{(2+2), \mathscr{E} \Rightarrow \hat{2} + \hat{2} \Rightarrow \hat{4}}}{(2+(2+2)), \mathscr{E} \Rightarrow \hat{2} + \hat{4} \Rightarrow \hat{6}}}{((1+2)+(2+(2+2))), \mathscr{E} \Rightarrow \hat{3} + \hat{6} \Rightarrow \hat{9}}
        $
    \end{center}
    \item Da el juicio de tipo para la siguiente expresion, la cual esta implementada en el lenguaje Racket.
    \begin{lstlisting}[language=Lisp]
(define fib
    (lambda (n)
        (if (<= n 1)
            n
            (+ (fib (- n 1)) (fib (- n 2))))))
    \end{lstlisting}

    \textbf{Solucion:}\\
    Empezamos a construir el juicio de tipos para la expresion dada.\\
    Nota: Abreviaremos num como el tipo number para facilitar la escritura del juicio y también lo desarrollaremos por partes por el espacio que requiere.\\
    Primero, nos fijamos en la expresión completa y empezamos a construir el juicio de tipos a partir de esta misma.\\

    \begin{center}
        \includegraphics[width=0.85\textwidth]{1.jpg}
    \end{center}

    Ahora bien, nos fijamos en la expresión del \texttt{if} y empezamos a construir el juicio de tipos para esta.\\
    Sea el entorno de tipos para la expresión del \texttt{if} el cual nos genera 3 ramas para el juicio de tipos, una por cada una de las expresiones que componen el \texttt{if}.\\
    La primera rama: 
    \begin{center}
        \includegraphics[width=0.85\textwidth]{2.jpg}
    \end{center}
    La segunda rama:
    \begin{center}
        \includegraphics[width=0.85\textwidth]{3.jpg}
    \end{center}
    La tercera y ultima rama:
    \begin{center}
        \includegraphics[width=0.85\textwidth]{4.jpg}
    \end{center}
    Por tanto el juicio de tipos es el árbol mostrado en las anteriores imágenes mostrado por partes.\\




    \item Realiza la inferencia de tipos de la siguiente expresión, exponiendo de manera explícita los tipos que recibe y regresa la expresión al finalizar la inferencia.

    En este ejercicio, se sustituyó el argumento de null? (nlst) por (lst).
    \begin{lstlisting}[language=Lisp]
(define foo
    (lambda (lst item)
        (cond
            ((null? lst) nempty)
            ((equal? item (nfirst lst)) (nrest lst))
            (else (ncons (nfirst lst) (foo (nrest lst) item))))))
    \end{lstlisting}
    Utilizamos el siguiente etiquetado de expresiones para facilitar la inferencia de tipos:\\
    \begin{itemize}
        \item \fbox{1} = \texttt{(lambda (lst item)
        (cond
            ((null? nlst) nempty)
            ((equal? item (nfirst lst)) (nrest lst))
            (else (ncons (nfirst lst) (foo (nrest lst) item)))))}
        \item \fbox{2} = \texttt{(cond
            ((null? nlst) nempty)
            ((equal? item (nfirst lst)) (nrest lst))
            (else (ncons (nfirst lst) (foo (nrest lst) item))))}
        \item \fbox{3} = \texttt{(null? lst)}
        \item \fbox{4} = \texttt{nempty}
        \item \fbox{5} = \texttt{(equal? item (nfirst lst))}
        \item \fbox{6} = \texttt{(nfirst lst)}
        \item \fbox{7} = \texttt{(nrest lst)}
        \item \fbox{8} = \texttt{else}
        \item \fbox{9} = \texttt{(ncons (nfirst lst) (foo (nrest lst) item))}
        \item \fbox{10} = \texttt{(nfirst lst)}
        \item \fbox{11} = \texttt{(foo (nrest lst) item)}
        \item \fbox{12} = \texttt{(nrest lst)}
    \end{itemize}
    Asumiendo que como en clase, los predicados \texttt{equal?} toma tipos \textit{number} y \texttt{nfirst} devuelva tipo \textit{number},
    la inferencia de tipos queda de la siguiente manera:
    \begin{itemize}
        \item \textlbrackdbl \ \fbox{1} \textrbrackdbl \ = (\textlbrackdbl \ lst \textrbrackdbl $\times$ \textlbrackdbl \ item \textrbrackdbl) $\rightarrow$ \textlbrackdbl \ \fbox{2} \textrbrackdbl\\
        Ya que se trata de una lambda y \fbox{2} es el cuerpo de esta.
        
        \item \textlbrackdbl \ \fbox{2} \textrbrackdbl \ = \textlbrackdbl \ \fbox{3} \textrbrackdbl $\rightarrow$ \textlbrackdbl \ \fbox{4} \textrbrackdbl \ \textbar \ \textlbrackdbl \ \fbox{5} \textrbrackdbl $\rightarrow$ \textlbrackdbl \ \fbox{7} \textrbrackdbl \ \textbar \ \textlbrackdbl \ \fbox{8} \textrbrackdbl $\rightarrow$ \textlbrackdbl \ \fbox{9} \textrbrackdbl, donde
        \textlbrackdbl \ \fbox{4} \textrbrackdbl \ = \textlbrackdbl \ \fbox{7} \textrbrackdbl \ = \textlbrackdbl \ \fbox{9} \textrbrackdbl \\
        Ya que se trata de un cond con tres posibles casos diferentes.

        \item \textlbrackdbl \ \fbox{3} \textrbrackdbl = \textlbrackdbl \ \texttt{null? lst} \textrbrackdbl = \textit{boolean} con \textlbrackdbl \ \texttt{lst} \textrbrackdbl = \textit{nlist}

        \item \textlbrackdbl \ \fbox{4} \textrbrackdbl = \textlbrackdbl \ \texttt{nempty} \textrbrackdbl = \textit{nlist}

        \item \textlbrackdbl \ \fbox{5} \textrbrackdbl = \textlbrackdbl \ \texttt{equal? item (nfirst lst)} \textrbrackdbl = \textit{boolean} con \textlbrackdbl \ \texttt{item} \textrbrackdbl = \textit{number} y
        \textlbrackdbl \ \fbox{6} \textrbrackdbl = \textit{number}

        \item \textlbrackdbl \ \fbox{6} \textrbrackdbl = \textlbrackdbl \ \texttt{(nfirst lst)} \textrbrackdbl = \textit{number} con \textlbrackdbl \ \texttt{lst} \textrbrackdbl = \textit{nlist}

        \item \textlbrackdbl \ \fbox{7} \textrbrackdbl (i.e.) \textlbrackdbl \ \texttt{nrest lst} \textrbrackdbl = \textit{nlist} con \textlbrackdbl \ \texttt{lst} \textrbrackdbl = \textit{nlist}

        \item \textlbrackdbl \ \fbox{8} \textrbrackdbl = \textlbrackdbl \ \texttt{else} \textrbrackdbl = \textit{boolean} ya que según el P.L.A.I, true y else en este caso del cond son equivalentes (se suele usar true para hacer este tipo de análisis en vez de else).

        \item \textlbrackdbl \ \fbox{9} \textrbrackdbl = \textit{nlist} con \textlbrackdbl \ \fbox{10} \textrbrackdbl (i.e.) \textlbrackdbl \ \texttt{nfirst lst} \textrbrackdbl = \textit{number} y
        \textlbrackdbl \ \fbox{11} \textrbrackdbl = \textit{nlist}
        Ya que se trata de la aplicación de un ncons con sus dos argumentos.

        \item \textlbrackdbl \ \fbox{10} \textrbrackdbl (i.e) \textlbrackdbl \ \texttt{nfirst lst} \textrbrackdbl = \textit{number} con \textlbrackdbl \ \texttt{lst} \textrbrackdbl = \textit{nlist}

        \item \textlbrackdbl \ \fbox{11} \textrbrackdbl se tiene la restricción \textlbrackdbl \ \texttt{foo} \textrbrackdbl = (\textlbrackdbl \ \fbox{12} \textrbrackdbl $\times$ \textlbrackdbl \ \texttt{item} \textrbrackdbl) $\rightarrow$
        \textlbrackdbl \ \texttt{foo} \fbox{12} \texttt{item} \textrbrackdbl

        \item \textlbrackdbl \ \fbox{12} \textrbrackdbl (i.e) \textlbrackdbl \ \texttt{nrest lst} \textrbrackdbl = \textit{nlist} con \textlbrackdbl \ \texttt{lst} \textrbrackdbl = \textit{nlist}

        Por lo que tenemos que \textlbrackdbl \ \texttt{lst} \textrbrackdbl = \textit{nlist}, así como \textlbrackdbl \ \fbox{12} \textrbrackdbl = \textit{nlist}. Así mismo, se infiere por \fbox{5} que \textlbrackdbl \ \texttt{item} \textrbrackdbl = \textit{number}.\\
        Por lo que el tipo de \texttt{foo} es:
        \textlbrackdbl \ \texttt{foo} \textrbrackdbl = ( \textit{nlist}  $\times$ \textit{number}) $\rightarrow$
        \textlbrackdbl \ \texttt{foo} \textit{nlist} \textit{number} \textrbrackdbl.
        Por otro lado, teniendo en cuenta, el retorno de los casos del cond, y con la restricción impuesta por \texttt{cons} (\fbox{9}), se tiene que \texttt{foo} devuelve algo de tipo \textit{nlist}, por lo que tenemos que el tipo de la función foo es:\\
        $nlist \times number \rightarrow nlist$
    \end{itemize}
    
    \item Utilizando el algoritmo de unificación visto en clase en la siguiente expresión:\\
    ((lambda (y) (* y (+ 0 0))) 1)\\
    Obtenemos las sub-expresiones y las nombramos:\\
    \fbox{1} ( \fbox{2} (lambda (y) \fbox{3} (* y \fbox{4} (+ \fbox{5} 0 \fbox{6} 0))) \fbox{7} 1)\\
    Ahora generamos las restricciones de tipos a partir de las sub-expresiones obtenidas:
    \begin{itemize}
        \item \textlbrackdbl \ \fbox{2} \textrbrackdbl \ = \textlbrackdbl \ \fbox{7} \textrbrackdbl \ $\rightarrow$ \textlbrackdbl \ \fbox{1} \textrbrackdbl
        \item \textlbrackdbl \ \fbox{2} \textrbrackdbl \ = \textlbrackdbl \ y \textrbrackdbl \ $\rightarrow$ \textlbrackdbl \ \fbox{3} \textrbrackdbl
        \item \textlbrackdbl \ \fbox{3} \textrbrackdbl \ = Number
        \item \textlbrackdbl \ y \textrbrackdbl \; \ = Number
        \item \textlbrackdbl \ \fbox{4} \textrbrackdbl \ = Number
        \item \textlbrackdbl \ \fbox{5} \textrbrackdbl \ = Number
        \item \textlbrackdbl \ \fbox{6} \textrbrackdbl \ = Number
        \item \textlbrackdbl \ \fbox{7} \textrbrackdbl \ = Number
    \end{itemize}
    Ejecutamos el algoritmo de unificación:
    \begin{longtable}{|c|l|l|}
    \hline
    Acción & \multicolumn{1}{c|}{Stack} & \multicolumn{1}{c|}{Sustitución} \\ \hline
    \endfirsthead
    \hline
    Acción & \multicolumn{1}{c|}{Stack} & \multicolumn{1}{c|}{Sustitución} \\ \hline
    \endhead
    Inicio & \textlbrackdbl \ \fbox{2} \textrbrackdbl \ = \textlbrackdbl \ \fbox{7} \textrbrackdbl $\rightarrow$ \textlbrackdbl \ \fbox{1} \textrbrackdbl & Vacío \\
           & \textlbrackdbl \ \fbox{2} \textrbrackdbl \ = \textlbrackdbl \ y \textrbrackdbl $\rightarrow$ \textlbrackdbl \ \fbox{3} \textrbrackdbl & \\
           & \textlbrackdbl \ \fbox{3} \textrbrackdbl \ = Number & \\
           & \textlbrackdbl \ y \textrbrackdbl \; \ = Number & \\
           & \textlbrackdbl \ \fbox{4} \textrbrackdbl \ = Number & \\
           & \textlbrackdbl \ \fbox{5} \textrbrackdbl \ = Number & \\
           & \textlbrackdbl \ \fbox{6} \textrbrackdbl \ = Number & \\
           & \textlbrackdbl \ \fbox{7} \textrbrackdbl \ = Number & \\ \hline
    Paso 3 & \textlbrackdbl \ \fbox{7} \textrbrackdbl \ $\rightarrow$ \textlbrackdbl \ \fbox{1} \textrbrackdbl = \textlbrackdbl \ y \textrbrackdbl \ $\rightarrow$ \textlbrackdbl \ \fbox{3} \textrbrackdbl & \textlbrackdbl \ \fbox{2} \textrbrackdbl \ $\mapsto$ (\textlbrackdbl \ \fbox{7} \textrbrackdbl \ $\rightarrow$ \textlbrackdbl \ \fbox{1} \textrbrackdbl) \\
           & \textlbrackdbl \ \fbox{3} \textrbrackdbl \ = Number & \\
           & \textlbrackdbl \ y \textrbrackdbl \; \ = Number & \\
           & \textlbrackdbl \ \fbox{4} \textrbrackdbl \ = Number & \\
           & \textlbrackdbl \ \fbox{5} \textrbrackdbl \ = Number & \\
           & \textlbrackdbl \ \fbox{6} \textrbrackdbl \ = Number & \\
           & \textlbrackdbl \ \fbox{7} \textrbrackdbl \ = Number & \\ \hline
    Paso 5 & \textlbrackdbl \ \fbox{7} \textrbrackdbl \ = \textlbrackdbl \ y \textrbrackdbl & \textlbrackdbl \ \fbox{2} \textrbrackdbl \ $\mapsto$ (\textlbrackdbl \ \fbox{7} \textrbrackdbl \ $\rightarrow$ \textlbrackdbl \ \fbox{1} \textrbrackdbl) \\
           & \textlbrackdbl \ \fbox{1} \textrbrackdbl \ = \textlbrackdbl \ \fbox{3} \textrbrackdbl & \\
           & \textlbrackdbl \ \fbox{3} \textrbrackdbl \ = Number & \\
           & \textlbrackdbl \ y \textrbrackdbl \; \ = Number & \\
           & \textlbrackdbl \ \fbox{4} \textrbrackdbl \ = Number & \\
           & \textlbrackdbl \ \fbox{5} \textrbrackdbl \ = Number & \\
           & \textlbrackdbl \ \fbox{6} \textrbrackdbl \ = Number & \\
           & \textlbrackdbl \ \fbox{7} \textrbrackdbl \ = Number & \\ \hline
    Paso 3 & \textlbrackdbl \ \fbox{1} \textrbrackdbl \ = \textlbrackdbl \ \fbox{3} \textrbrackdbl & \textlbrackdbl \ \fbox{2} \textrbrackdbl \ $\mapsto$ (\textlbrackdbl \ y \textrbrackdbl \ $\rightarrow$ \textlbrackdbl \ \fbox{1} \textrbrackdbl) \\
           & \textlbrackdbl \ \fbox{3} \textrbrackdbl \ = Number & \textlbrackdbl \ \fbox{7} \textrbrackdbl \ $\mapsto$ \textlbrackdbl \ y \textrbrackdbl \\
           & \textlbrackdbl \ y \textrbrackdbl \; \ = Number & \\
           & \textlbrackdbl \ \fbox{4} \textrbrackdbl \ = Number & \\
           & \textlbrackdbl \ \fbox{5} \textrbrackdbl \ = Number & \\
           & \textlbrackdbl \ \fbox{6} \textrbrackdbl \ = Number & \\
           & \textlbrackdbl \ y \textrbrackdbl \; \ = Number & \\ \hline
    Paso 3 & \textlbrackdbl \ \fbox{3} \textrbrackdbl \ = Number & \textlbrackdbl \ \fbox{2} \textrbrackdbl \ $\mapsto$ (\textlbrackdbl \ y \textrbrackdbl \ $\rightarrow$ \textlbrackdbl \ \fbox{3} \textrbrackdbl) \\
           & \textlbrackdbl \ y \textrbrackdbl \; \ = Number & \textlbrackdbl \ \fbox{7} \textrbrackdbl \ $\mapsto$ \textlbrackdbl \ y \textrbrackdbl \\
           & \textlbrackdbl \ \fbox{4} \textrbrackdbl \ = Number & \textlbrackdbl \ \fbox{1} \textrbrackdbl \ $\mapsto$ \textlbrackdbl \ \fbox{3} \textrbrackdbl \\
           & \textlbrackdbl \ \fbox{5} \textrbrackdbl \ = Number & \\
           & \textlbrackdbl \ \fbox{6} \textrbrackdbl \ = Number & \\
           & \textlbrackdbl \ y \textrbrackdbl \; \ = Number & \\ \hline
    Paso 3 & \textlbrackdbl \ y \textrbrackdbl \; \ = Number & \textlbrackdbl \ \fbox{2} \textrbrackdbl \ $\mapsto$ (\textlbrackdbl \ y \textrbrackdbl \ $\rightarrow$ Number) \\
           & \textlbrackdbl \ \fbox{4} \textrbrackdbl \ = Number & \textlbrackdbl \ \fbox{7} \textrbrackdbl \ $\mapsto$ \textlbrackdbl \ y \textrbrackdbl \\
           & \textlbrackdbl \ \fbox{5} \textrbrackdbl \ = Number & \textlbrackdbl \ \fbox{1} \textrbrackdbl \ $\mapsto$ Number \\
           & \textlbrackdbl \ \fbox{6} \textrbrackdbl \ = Number & \textlbrackdbl \ \fbox{3} \textrbrackdbl \ $\mapsto$ Number \\
           & \textlbrackdbl \ y \textrbrackdbl \; \ = Number & \\ \hline
    Paso 3 & \textlbrackdbl \ \fbox{4} \textrbrackdbl \ = Number & \textlbrackdbl \ \fbox{2} \textrbrackdbl \ $\mapsto$ (Number \ $\rightarrow$ Number) \\
           & \textlbrackdbl \ \fbox{5} \textrbrackdbl \ = Number & \textlbrackdbl \ \fbox{7} \textrbrackdbl \ $\mapsto$ Number \\
           & \textlbrackdbl \ \fbox{6} \textrbrackdbl \ = Number & \textlbrackdbl \ \fbox{1} \textrbrackdbl \ $\mapsto$ Number \\
           & Number = Number & \textlbrackdbl \ \fbox{3} \textrbrackdbl \ $\mapsto$ Number \\ 
           &  & \textlbrackdbl \ y \textrbrackdbl \; \ $\mapsto$ Number \\ \hline
    Paso 3 & \textlbrackdbl \ \fbox{5} \textrbrackdbl \ = Number & \textlbrackdbl \ \fbox{2} \textrbrackdbl \ $\mapsto$ (Number \ $\rightarrow$ Number) \\
           & \textlbrackdbl \ \fbox{6} \textrbrackdbl \ = Number & \textlbrackdbl \ \fbox{7} \textrbrackdbl \ $\mapsto$ Number \\
           & Number = Number & \textlbrackdbl \ \fbox{1} \textrbrackdbl \ $\mapsto$ Number \\ 
           &  & \textlbrackdbl \ \fbox{3} \textrbrackdbl \ $\mapsto$ Number \\
           &  & \textlbrackdbl \ y \textrbrackdbl \; \ $\mapsto$ Number \\
           &  & \textlbrackdbl \ \fbox{4} \textrbrackdbl \ $\mapsto$ Number \\ \hline
    Paso 3 & \textlbrackdbl \ \fbox{6} \textrbrackdbl \ = Number & \textlbrackdbl \ \fbox{2} \textrbrackdbl \ $\mapsto$ (Number \ $\rightarrow$ Number) \\
           & Number = Number & \textlbrackdbl \ \fbox{7} \textrbrackdbl \ $\mapsto$ Number \\
           &  & \textlbrackdbl \ \fbox{1} \textrbrackdbl \ $\mapsto$ Number \\ 
           &  & \textlbrackdbl \ \fbox{3} \textrbrackdbl \ $\mapsto$ Number \\
           &  & \textlbrackdbl \ y \textrbrackdbl \; \ $\mapsto$ Number \\
           &  & \textlbrackdbl \ \fbox{4} \textrbrackdbl \ $\mapsto$ Number \\
           &  & \textlbrackdbl \ \fbox{5} \textrbrackdbl \ $\mapsto$ Number \\ \hline
    Paso 3 & Number = Number & \textlbrackdbl \ \fbox{2} \textrbrackdbl \ $\mapsto$ (Number \ $\rightarrow$ Number) \\
           &  & \textlbrackdbl \ \fbox{7} \textrbrackdbl \ $\mapsto$ Number \\
           &  & \textlbrackdbl \ \fbox{1} \textrbrackdbl \ $\mapsto$ Number \\ 
           &  & \textlbrackdbl \ \fbox{3} \textrbrackdbl \ $\mapsto$ Number \\
           &  & \textlbrackdbl \ y \textrbrackdbl \; \ $\mapsto$ Number \\
           &  & \textlbrackdbl \ \fbox{4} \textrbrackdbl \ $\mapsto$ Number \\
           &  & \textlbrackdbl \ \fbox{5} \textrbrackdbl \ $\mapsto$ Number \\
           &  & \textlbrackdbl \ \fbox{6} \textrbrackdbl \ $\mapsto$ Number \\ \hline
    Paso 1 & Vacío & \textlbrackdbl \ \fbox{2} \textrbrackdbl \ $\mapsto$ (Number \ $\rightarrow$ Number) \\
           &  & \textlbrackdbl \ \fbox{7} \textrbrackdbl \ $\mapsto$ Number \\
           &  & \textlbrackdbl \ \fbox{1} \textrbrackdbl \ $\mapsto$ Number \\ 
           &  & \textlbrackdbl \ \fbox{3} \textrbrackdbl \ $\mapsto$ Number \\
           &  & \textlbrackdbl \ y \textrbrackdbl \; \ $\mapsto$ Number \\
           &  & \textlbrackdbl \ \fbox{4} \textrbrackdbl \ $\mapsto$ Number \\
           &  & \textlbrackdbl \ \fbox{5} \textrbrackdbl \ $\mapsto$ Number \\
           &  & \textlbrackdbl \ \fbox{6} \textrbrackdbl \ $\mapsto$ Number \\ \hline
    \end{longtable}
    Así queda hecho el algoritmo de unificación en la expresión dada de acuerdo al algoritmo de unificación que viene justo aquí en la tarea.
\end{enumerate}
\subsection*{Algoritmo de Unificación}
\begin{enumerate}
    \item Si X e Y son constante idénticas, no se hace nada.
    \item Si X e Y son identificadores idénticos, no se hace nada.
    \item Si X es un identificador, reemplaza todas las ocurrencias de X por Y tanto en el stack como en la sustitución, y añade X $\rightarrow$ Y en la sustitución.
    \item Si Y es un identificador, reemplaza todas las ocurrencias de Y por X tanto en el stack como en la sustitución, y añade Y $\rightarrow$ X en la sustitución.
    \item Si X es de la forma C($X_1$, $X_2$ ,...., $X_n$) para algun constructor C, e Y es de la forma C($Y_1$, $Y_2$, ...., $Y_n$) (i.e. tienen el mismo constructor), entonces agrega $X_i = Y_i$ para toda $1 \leq i \leq n$ en el stack.
    \item En cualquier otro caso, X e Y no se unifican y se reporta un error.
\end{enumerate}
\end{document}
