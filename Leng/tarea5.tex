\documentclass[11pt]{article}
\usepackage[spanish]{babel}
\usepackage[utf8]{inputenc}
\usepackage{amsmath, amssymb}
\usepackage{geometry}
\usepackage{setspace}
\usepackage{enumitem}
\usepackage{graphicx}
\usepackage{xurl}
\usepackage{listings}
\usepackage{tikz}
\usepackage{xcolor}
\usetikzlibrary{decorations.pathreplacing, calc}

\geometry{letterpaper, margin=2cm}

\begin{document}
\begin{titlepage}
    \centering
    \vspace{2cm}
    {\includegraphics[height=3.2cm]{../logo_unam.png}}
    \hfill
    {\includegraphics[height=3.2cm]{../logo_fc.png}\par}
    \vspace{1cm}
    {\bfseries\LARGE UNIVERSIDAD NACIONAL AUTÓNOMA DE MÉXICO \par}
    \vspace{0.7cm}
    {\scshape\Large FACULTAD DE CIENCIAS \par}
    \vspace{1cm}
    {\itshape\Large Lenguajes de programación \par}
    \vspace{0.5cm}
    {\itshape\Large Semestre 2026-1 \par}
    \vspace{2cm}
    {\scshape\Huge Tarea 4 \par}
    \vspace{1cm}
    {\itshape\Large Fecha de entrega: 20 de octubre de 2025 \par}
    \vspace{2cm}
    {\Large Autores: \par}
    \vspace{0.4cm}
    {\Large Escobar Gonzalez Isaac Giovani \hspace{1cm} 321336400 \par}
    {\Large Garduño Escobar Kevin Jonathan \hspace{0.5cm} 321070629 \par}
    {\Large Zaldivar Alanis Rodrigo \hspace{2.75cm} 424029605 \par}
\end{titlepage}
\section*{Instrucciones}
\noindent Resolver los siguientes ejercicios de forma clara y ordenada de acuerdo a los lineamientos de entrega de tareas disponibles en la página del curso.\\
\section*{Ejercicios}

\begin{enumerate}[leftmargin=0.8cm]
    \item Utiliza el paso de parámetros que se indica para evaluar la siguiente expresión.\\
    \begin{enumerate}
        \item Paso de parámetros por valor.
        \item Paro de parámetros por referencia.\\
    \end{enumerate}
    \begin{lstlisting}
{multi-with {{a 8} {b -8}
    {swap {fun {x y}
        {multi-with {{tmp x}} {seqn {set x y} 
                                    {set y tmp}}}}}}
    
    {seqn {swap a b}
            {-a {+ b a}}}}
    \end{lstlisting}
    \item Utiliza el paso de parámetros que se indica para evaluar la siguiente expresión.
    \begin{enumerate}
        \item Paso de parámetros por nombre.
        \item Paro de parámetros por necesidad.\\
    \end{enumerate}
    \begin{lstlisting}
{multi-with {{a 1
    foo {fun {x}
        {seqn {set a {+ {* x -2} a}} a}}}}
            {{fun {y}{+ y {- y y}}}{foo 3}}}
    \end{lstlisting}
    \item Define las siguientes funciones de manera recursiva.
    \begin{enumerate}
        \item La función flatMap cuya entrada es una lista de listas y la salida una lista, el objetivo de la función es aplanar una lista, es decir, unir las listas que estén adentro de una lista en una sola. Por ejemplo:\\
        $>$ (\texttt{flatMap} '((3 5 8) (5 2 1 2 2 0 3)))\\
        $>$ '(3 5 8 5 2 1 2 2 0 3)\\
        La función quedaría definida como:
        \begin{lstlisting}
(define (flatMap list)
    (if (empty? list)
        '()
        (append (first list) (flatMap (cdr list)))))
        \end{lstlisting}
        \item La función Tribonacci cuya entrada es un número y la salida el respectivo resultado de Tribonacci de ese número. Recordando que el Tribonacci se define formalmente como:
        \[
            T(n) = \left\{ \begin{array}{ll}
                0, & \text{si } n= 0,\\
                1, & \text{si } n= 1,\\
                1, & \text{si } n= 2,\\
                T(n-1) + T(n-2) + T(n-3), & \text{si } n > 2.
            \end{array}\right.
        \]
        Es decir en Tribonacci se suman los tres números anteriores para obtener el número a calcular. Por ejemplo:\\
        $>$ (\texttt{Tribonacci} 5)\\
        $>$ 7\\
        La función quedaría definida como:
        \begin{lstlisting}
(define (Tribonacci n)
  (cond
    [(zero? n) 0]
    [(or (= n 2) (= n 1)) 1]
    [else (+ (Tribonacci (sub1 n)) 
          (+ (Tribonacci (- n 2)) (Tribonacci (- n 3))))]))
        \end{lstlisting}
    \end{enumerate}
    \item A partir de las funciones realizadas en el Ejercicio 3, muestra como se verían los registros de activación generados con los siguientes ejemplares.
    \begin{enumerate}
        \item flatMap\\
        $>$ (\texttt{flatMap} '((1 2 3) (1 2))) 
        \item Tribonacci\\
        $>$ (\texttt{Tribonacci} 4)\\
    \end{enumerate}
    \item Usando la técnica de recursión de cola optimiza las funciones del Ejercicio 3. Toda función auxiliar que utilices debe ser optimizada también utilizando esta misma técnica.\\
    \begin{enumerate}
        \item flatMap con recursión de cola.
        \begin{lstlisting}
(define (flatMap list)
  (flatMapAux list '()))
(define (flatMapAux list acc)
  (if (empty? list)
      acc
      (flatMapAux (cdr list) (append acc (first list)))))
        \end{lstlisting}
        \item Tribonacci con recursión de cola.
        \begin{lstlisting}
(define (Tribonacci n)
  (TribonacciAux n 0 1 1))
(define (TribonacciAux n acc1 acc2 acc3)
  (cond
    [(zero? n) acc1]
    [(= n 1) acc2]
    [(= n 2) acc3]
    [else (TribonacciAux (sub1 n) acc2 acc3 (+ acc1 (+ acc2 acc3)))]))
        \end{lstlisting}
    \end{enumerate}
    \item A partir de las funciones optimizadas realizadas en el Ejercicio 3, muestra como se verían los registros de activación generados con los siguientes ejemplares.
    \begin{enumerate}
        \item flatMap\\
        $>$ (\texttt{flatMap} '((1 2 3) (1 2)))
        \item Tribonacci\\
        $>$ (\texttt{Tribonacci} 4)\\
    \end{enumerate}
\end{enumerate}

\end{document}
