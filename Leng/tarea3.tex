\documentclass[11pt]{article}
\usepackage[spanish]{babel}
\usepackage[utf8]{inputenc}
\usepackage{amsmath, amssymb}
\usepackage{geometry}
\usepackage{setspace}
\usepackage{enumitem}
\usepackage{graphicx}
\usepackage{xurl}
\usepackage{listings}

\geometry{letterpaper, margin=2cm}

\begin{document}
\begin{titlepage}
    \centering
    \vspace{2cm}
    {\includegraphics[height=3.2cm]{../logo_unam.png}}
    \hfill
    {\includegraphics[height=3.2cm]{../logo_fc.png}\par}
    \vspace{1cm}
    {\bfseries\LARGE UNIVERSIDAD NACIONAL AUTÓNOMA DE MÉXICO \par}
    \vspace{0.7cm}
    {\scshape\Large FACULTAD DE CIENCIAS \par}
    \vspace{1cm}
    {\itshape\Large Lenguajes de programación \par}
    \vspace{0.5cm}
    {\itshape\Large Semestre 2026-1 \par}
    \vspace{2cm}
    {\scshape\Huge Tarea 3 \par}
    \vspace{1cm}
    {\itshape\Large Fecha de entrega: 19 de septiembre de 2025 \par}
    \vspace{2cm}
    {\Large Autores: \par}
    \vspace{0.4cm}
    {\Large Escobar Gonzalez Isaac Giovani \hspace{1cm} 321336400 \par}
    {\Large Garduño Escobar Kevin Jonathan \hspace{0.5cm} 321070629 \par}
    {\Large Zaldivar Alanis Rodrigo \hspace{2.75cm} 424029605 \par}
\end{titlepage}
\section*{Instrucciones}
\noindent Resolver los siguientes ejercicios de forma clara y ordenada de acuerdo a los lineamientos de entrega de tareas disponibles en la página del curso.\\
\section*{Ejercicios}

\begin{enumerate}[leftmargin=0.8cm]

    \item Evalúa las siguientes expresiones usando el régimen de evaluación especificado en cada inciso. En cada inciso coloca el ambiente \textbf{en forma de lista}, con el cual se evaluó la expresión.
    \begin{lstlisting}
{with {x {- 13 {* 5 8}}}
    {with {y {- x x}}
        {with {foo {fun {a} {- x y}}}
            {with {x {+ (-13) {- 3 6}}}
                {with {y {+ x x}}
                    {foo -3}}}}}}
    \end{lstlisting}
    \begin{itemize}
        \item Alcance estático. \textbf{(1 punto)}
        \item Alcance dinámico. \textbf{(1 punto)}
    \end{itemize}
    \item Evalúa las siguientes expresiones usando el régimen de evaluación y alcance especificado en cada inciso. En cada inciso coloca el ambiente \textbf{en forma de pila}, con el cual se evaluó la expresión.
    \begin{lstlisting}
{with {a {- 5 5}}
    {with {b {- a a}}
        {with {foo {fun {x} {- a b}}}
            {with {a {+ (-3) (-3)}}
                {with {b {+ a a}}
                    {foo 1}}}}}}
    \end{lstlisting}
    \begin{itemize}
        \item Evaluación glotona y alcance estático. \textbf{(2 puntos)}
        \item Evaluación glotona y alcance dinámico. \textbf{(2 puntos)}
        \item Evaluación perezosa y alcance estático. \textbf{(2 puntos)}
        \item Evaluación perezosa y alcance dinámico. \textbf{(2 puntos)}
    \end{itemize}
\end{enumerate}

\end{document}
