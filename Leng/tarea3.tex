\documentclass[11pt]{article}
\usepackage[spanish]{babel}
\usepackage[utf8]{inputenc}
\usepackage{amsmath, amssymb}
\usepackage{geometry}
\usepackage{setspace}
\usepackage{enumitem}
\usepackage{graphicx}
\usepackage{xurl}
\usepackage{listings}
\usepackage{tikz}
\usepackage{xcolor}
\usetikzlibrary{decorations.pathreplacing, calc}

\geometry{letterpaper, margin=2cm}

\begin{document}
\begin{titlepage}
    \centering
    \vspace{2cm}
    {\includegraphics[height=3.2cm]{../logo_unam.png}}
    \hfill
    {\includegraphics[height=3.2cm]{../logo_fc.png}\par}
    \vspace{1cm}
    {\bfseries\LARGE UNIVERSIDAD NACIONAL AUTÓNOMA DE MÉXICO \par}
    \vspace{0.7cm}
    {\scshape\Large FACULTAD DE CIENCIAS \par}
    \vspace{1cm}
    {\itshape\Large Lenguajes de programación \par}
    \vspace{0.5cm}
    {\itshape\Large Semestre 2026-1 \par}
    \vspace{2cm}
    {\scshape\Huge Tarea 3 \par}
    \vspace{1cm}
    {\itshape\Large Fecha de entrega: 19 de septiembre de 2025 \par}
    \vspace{2cm}
    {\Large Autores: \par}
    \vspace{0.4cm}
    {\Large Escobar Gonzalez Isaac Giovani \hspace{1cm} 321336400 \par}
    {\Large Garduño Escobar Kevin Jonathan \hspace{0.5cm} 321070629 \par}
    {\Large Zaldivar Alanis Rodrigo \hspace{2.75cm} 424029605 \par}
\end{titlepage}
\section*{Instrucciones}
\noindent Resolver los siguientes ejercicios de forma clara y ordenada de acuerdo a los lineamientos de entrega de tareas disponibles en la página del curso.\\
\section*{Ejercicios}

\begin{enumerate}[leftmargin=0.8cm]

    \item Evalúa las siguientes expresiones usando el régimen de evaluación especificado en cada inciso. En cada inciso coloca el ambiente \textbf{en forma de lista}, con el cual se evaluó la expresión.
    \begin{lstlisting}
{with {x {- 13 {* 5 8}}}
    {with {y {- x x}}
        {with {foo {fun {a} {- x y}}}
            {with {x {+ (-13) {- 3 6}}}
                {with {y {+ x x}}
                    {foo -3}}}}}}
    \end{lstlisting}
    \begin{itemize}
        \item Alcance estático. \textbf{(1 punto)}\\
            Empezamos construyendo la lista del ambiente:\\
            \hspace*{-0.5cm}
            \begin{tikzpicture}
                \node (list) at (0,0) {( (y \, (+ x x)) \, (x \, (+ (-13) (- 3 6))) \, (foo \, (fun (a) (- x y))) \, (y \, (- x x)) \, (x \, (- 13 (* 5 8))) )};
                \draw[red, thick, decoration={brace, mirror}, decorate]
                    ($(list.south west)+(-0.1,0)$) -- ($(list.south east)+(0.1,0)$);
                \node[red] at ($($(list.south west)+(-0.2,0)$)!0.5!($(list.south east)+(0.2,0)$)$) [below=2pt] {\large Env};
            \end{tikzpicture}\\
            Ahora evaluamos el cuerpo del with que esta más adentro, es decir, la expresión \texttt{(foo -3)}. Para esto, buscamos el valor de \texttt{foo} en el ambiente y obtenemos la función \texttt{(fun (a) (- x y))}.\\ 
            Dado que el parámetro de la función \texttt{foo} es \texttt{a} y el argumento para la aplicación de la función es \texttt{-3}, entonces, añadimos el par \texttt{(a -3)} al ambiente, justo después de la función \texttt{foo} dado que es alcance estático.\\
            Ahora si podemos evaluar el cuerpo de la función \texttt{foo} que es \texttt{(- x y)}. Para esto, buscamos el valor de \texttt{x} en el ambiente y obtenemos \texttt{(- 13 (* 5 8))}. Evaluamos esta expresión y obtenemos que \texttt{x = (- 13 40) = -27}.\\
            Entonces buscamos el valor de \texttt{y} en el ambiente y obtenemos \texttt{(- x x)}. Evaluamos esta expresión usando el valor de \texttt{x} que acabamos de obtener, y nos queda que, \texttt{y = (- -27 -27) = 0}.\\
            Y por último, evaluamos la expresión \texttt{(- x y)} usando los valores que obtuvimos antes: \texttt{(- -27 0) = -27}. Por lo tanto, el resultado final es \textbf{-27}.\\
            El ambiente final quedo:\\
            \hspace*{-1cm}
            \begin{tikzpicture}
                \node (list) at (0,0) {( (y \, (+ x x)) \, (x \, (+ (-13) (- 3 6))) \, (foo \, (fun (a) (- x y))) \, (a \, -3) \, (y \, (- x x)) \, (x \, (- 13 (* 5 8))) )};
                \draw[red, thick, decoration={brace, mirror}, decorate]
                    ($(list.south west)+(-0.1,0)$) -- ($(list.south east)+(0.1,0)$);
                \node[red] at ($($(list.south west)+(-0.2,0)$)!0.5!($(list.south east)+(0.2,0)$)$) [below=2pt] {\large Env};
            \end{tikzpicture}
        \item Alcance dinámico. \textbf{(1 punto)}\\
            Como en el caso anterior, empezamos construyendo la lista del ambiente:\\
            \hspace*{-0.5cm}
            \begin{tikzpicture}
                \node (list) at (0,0) {( (y \, (+ x x)) \, (x \, (+ (-13) (- 3 6))) \, (foo \, (fun (a) (- x y))) \, (y \, (- x x)) \, (x \, (- 13 (* 5 8))) )};
                \draw[red, thick, decoration={brace, mirror}, decorate]
                    ($(list.south west)+(-0.1,0)$) -- ($(list.south east)+(0.1,0)$);
                \node[red] at ($($(list.south west)+(-0.2,0)$)!0.5!($(list.south east)+(0.2,0)$)$) [below=2pt] {\large Env};
            \end{tikzpicture}\\
            Ahora evaluamos la aplicación de la función \texttt{(foo -3)}. Para esto, buscamos el valor de \texttt{foo} en el ambiente y obtenemos la función \texttt{(fun (a) (- x y))}.\\
            Asignamos el valor \texttt{-3} al parámetro \texttt{a} de la función \texttt{foo}, pero en este caso, dado que es alcance dinámico, añadimos el par \texttt{(a -3)} al inicio del ambiente.\\
            Entonces, evaluamos el cuerpo de la función \texttt{foo} que es \texttt{(- x y)}. Para esto, buscamos el valor de \texttt{x} en el ambiente desde el inicio de la lista que es \texttt{(+ -13 (- 3 6))}.\\
            Así, \texttt{x = (+ -13 (- 3 6)) = -16}.\\
            Luego, buscamos el valor de \texttt{y} en el ambiente y obtenemos \texttt{(+ x x)}. Evaluamos esta expresión usando el valor de \texttt{x} que acabamos de obtener, y nos queda que, \texttt{y = (+ -16 -16) = -32}.\\
            Nos queda evaluar la función \texttt{(- x y)} usando los valores que obtuvimos antes: \texttt{(- -16 -32) = 16}. Por lo tanto, el resultado final es \textbf{16}.\\
            El ambiente final quedaría así:\\
            \hspace*{-1cm}
            \begin{tikzpicture}
                \node (list) at (0,0) {( (a \, -3) \, (y \, (+ x x)) \, (x \, (+ (-13) (- 3 6))) \, (foo \, (fun (a) (- x y))) \, (y \, (- x x)) \, (x \, (- 13 (* 5 8))) )};
                \draw[red, thick, decoration={brace, mirror}, decorate]
                    ($(list.south west)+(-0.1,0)$) -- ($(list.south east)+(0.1,0)$);
                \node[red] at ($($(list.south west)+(-0.2,0)$)!0.5!($(list.south east)+(0.2,0)$)$) [below=2pt] {\large Env};
            \end{tikzpicture}
    \end{itemize}
    \item Evalúa las siguientes expresiones usando el régimen de evaluación y alcance especificado en cada inciso. En cada inciso coloca el ambiente \textbf{en forma de pila}, con el cual se evaluó la expresión.
    \begin{lstlisting}
{with {a {- 5 5}}
    {with {b {- a a}}
        {with {foo {fun {x} {- a b}}}
            {with {a {+ (-3) (-3)}}
                {with {b {+ a a}}
                    {foo 1}}}}}}
    \end{lstlisting}
    \begin{itemize}
        \item Evaluación glotona y alcance estático. \textbf{(2 puntos)}
        \item Evaluación glotona y alcance dinámico. \textbf{(2 puntos)}
        \item Evaluación perezosa y alcance estático. \textbf{(2 puntos)}
        \item Evaluación perezosa y alcance dinámico. \textbf{(2 puntos)}
    \end{itemize}
\end{enumerate}

\end{document}
