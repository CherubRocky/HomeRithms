\documentclass[11pt]{article}
\usepackage[spanish]{babel}
\usepackage[utf8]{inputenc}
\usepackage{amsmath, amssymb}
\usepackage{geometry}
\usepackage{setspace}
\usepackage{enumitem}
\usepackage{graphicx}
\usepackage{xurl}
\usepackage{listings}
\usepackage{tikz}
\usepackage{xcolor}
\usetikzlibrary{decorations.pathreplacing, calc}

\geometry{letterpaper, margin=2cm}

\begin{document}
\begin{titlepage}
    \centering
    \vspace{2cm}
    {\includegraphics[height=3.2cm]{../logo_unam.png}}
    \hfill
    {\includegraphics[height=3.2cm]{../logo_fc.png}\par}
    \vspace{1cm}
    {\bfseries\LARGE UNIVERSIDAD NACIONAL AUTÓNOMA DE MÉXICO \par}
    \vspace{0.7cm}
    {\scshape\Large FACULTAD DE CIENCIAS \par}
    \vspace{1cm}
    {\itshape\Large Lenguajes de programación \par}
    \vspace{0.5cm}
    {\itshape\Large Semestre 2026-1 \par}
    \vspace{2cm}
    {\scshape\Huge Tarea 3 \par}
    \vspace{1cm}
    {\itshape\Large Fecha de entrega: 22 de septiembre de 2025 \par}
    \vspace{2cm}
    {\Large Autores: \par}
    \vspace{0.4cm}
    {\Large Escobar Gonzalez Isaac Giovani \hspace{1cm} 321336400 \par}
    {\Large Garduño Escobar Kevin Jonathan \hspace{0.5cm} 321070629 \par}
    {\Large Zaldivar Alanis Rodrigo \hspace{2.75cm} 424029605 \par}
\end{titlepage}
\section*{Instrucciones}
\noindent Resolver los siguientes ejercicios de forma clara y ordenada de acuerdo a los lineamientos de entrega de tareas disponibles en la página del curso.\\
\section*{Ejercicios}

\begin{enumerate}[leftmargin=0.8cm]

    \item Evalúa las siguientes expresiones usando el régimen de evaluación especificado en cada inciso. En cada inciso coloca el ambiente \textbf{en forma de lista}, con el cual se evaluó la expresión.
    \begin{lstlisting}
{with {x {- 13 {* 5 8}}}
    {with {y {- x x}}
        {with {foo {fun {a} {- x y}}}
            {with {x {+ (-13) {- 3 6}}}
                {with {y {+ x x}}
                    {foo -3}}}}}}
    \end{lstlisting}
    \begin{itemize}
        \item Alcance estático. \textbf{(1 punto)}\\
            Empezamos construyendo la lista del ambiente:\\
            \hspace*{-0.5cm}
            \begin{tikzpicture}
                \node (list) at (0,0) {( (y \, (+ x x)) \, (x \, (+ (-13) (- 3 6))) \, (foo \, (fun (a) (- x y))) \, (y \, (- x x)) \, (x \, (- 13 (* 5 8))) )};
                \draw[red, thick, decoration={brace, mirror}, decorate]
                    ($(list.south west)+(-0.1,0)$) -- ($(list.south east)+(0.1,0)$);
                \node[red] at ($($(list.south west)+(-0.2,0)$)!0.5!($(list.south east)+(0.2,0)$)$) [below=2pt] {\large Env};
            \end{tikzpicture}\\
            Ahora evaluamos el cuerpo del with que esta más adentro, es decir, la expresión \texttt{(foo -3)}. Para esto, buscamos el valor de \texttt{foo} en el ambiente y obtenemos la función \texttt{(fun (a) (- x y))}.\\
            Dado que el parámetro de la función \texttt{foo} es \texttt{a} y el argumento para la aplicación de la función es \texttt{-3}, entonces, añadimos el par \texttt{(a -3)} al ambiente, justo después de la función \texttt{foo} dado que es alcance estático.\\
            Ahora si podemos evaluar el cuerpo de la función \texttt{foo} que es \texttt{(- x y)}. Para esto, buscamos el valor de \texttt{x} en el ambiente y obtenemos \texttt{(- 13 (* 5 8))}. Evaluamos esta expresión y obtenemos que \texttt{x = (- 13 40) = -27}.\\
            Entonces buscamos el valor de \texttt{y} en el ambiente y obtenemos \texttt{(- x x)}. Evaluamos esta expresión usando el valor de \texttt{x} que acabamos de obtener, y nos queda que, \texttt{y = (- -27 -27) = 0}.\\
            Y por último, evaluamos la expresión \texttt{(- x y)} usando los valores que obtuvimos antes: \texttt{(- -27 0) = -27}. Por lo tanto, el resultado final es \textbf{-27}.\\
            El ambiente final quedo:\\
            \hspace*{-1cm}
            \begin{tikzpicture}
                \node (list) at (0,0) {( (y \, (+ x x)) \, (x \, (+ (-13) (- 3 6))) \, (foo \, (fun (a) (- x y))) \, (a \, -3) \, (y \, (- x x)) \, (x \, (- 13 (* 5 8))) )};
                \draw[red, thick, decoration={brace, mirror}, decorate]
                    ($(list.south west)+(-0.1,0)$) -- ($(list.south east)+(0.1,0)$);
                \node[red] at ($($(list.south west)+(-0.2,0)$)!0.5!($(list.south east)+(0.2,0)$)$) [below=2pt] {\large Env};
            \end{tikzpicture}
        \item Alcance dinámico. \textbf{(1 punto)}\\
            Como en el caso anterior, empezamos construyendo la lista del ambiente:\\
            \hspace*{-0.5cm}
            \begin{tikzpicture}
                \node (list) at (0,0) {( (y \, (+ x x)) \, (x \, (+ (-13) (- 3 6))) \, (foo \, (fun (a) (- x y))) \, (y \, (- x x)) \, (x \, (- 13 (* 5 8))) )};
                \draw[red, thick, decoration={brace, mirror}, decorate]
                    ($(list.south west)+(-0.1,0)$) -- ($(list.south east)+(0.1,0)$);
                \node[red] at ($($(list.south west)+(-0.2,0)$)!0.5!($(list.south east)+(0.2,0)$)$) [below=2pt] {\large Env};
            \end{tikzpicture}\\
            Ahora evaluamos la aplicación de la función \texttt{(foo -3)}. Para esto, buscamos el valor de \texttt{foo} en el ambiente y obtenemos la función \texttt{(fun (a) (- x y))}.\\
            Asignamos el valor \texttt{-3} al parámetro \texttt{a} de la función \texttt{foo}, pero en este caso, dado que es alcance dinámico, añadimos el par \texttt{(a -3)} al inicio del ambiente.\\
            Entonces, evaluamos el cuerpo de la función \texttt{foo} que es \texttt{(- x y)}. Para esto, buscamos el valor de \texttt{x} en el ambiente desde el inicio de la lista que es \texttt{(+ -13 (- 3 6))}.\\
            Así, \texttt{x = (+ -13 (- 3 6)) = -16}.\\
            Luego, buscamos el valor de \texttt{y} en el ambiente y obtenemos \texttt{(+ x x)}. Evaluamos esta expresión usando el valor de \texttt{x} que acabamos de obtener, y nos queda que, \texttt{y = (+ -16 -16) = -32}.\\
            Nos queda evaluar la función \texttt{(- x y)} usando los valores que obtuvimos antes: \texttt{(- -16 -32) = 16}. Por lo tanto, el resultado final es \textbf{16}.\\
            El ambiente final quedaría así:\\
            \hspace*{-1cm}
            \begin{tikzpicture}
                \node (list) at (0,0) {( (a \, -3) \, (y \, (+ x x)) \, (x \, (+ (-13) (- 3 6))) \, (foo \, (fun (a) (- x y))) \, (y \, (- x x)) \, (x \, (- 13 (* 5 8))) )};
                \draw[red, thick, decoration={brace, mirror}, decorate]
                    ($(list.south west)+(-0.1,0)$) -- ($(list.south east)+(0.1,0)$);
                \node[red] at ($($(list.south west)+(-0.2,0)$)!0.5!($(list.south east)+(0.2,0)$)$) [below=2pt] {\large Env};
            \end{tikzpicture}
    \end{itemize}
    \item Evalúa las siguientes expresiones usando el régimen de evaluación y alcance especificado en cada inciso. En cada inciso coloca el ambiente \textbf{en forma de pila}, con el cual se evaluó la expresión.
    \begin{lstlisting}
{with {a {- 5 5}}
    {with {b {- a a}}
        {with {foo {fun {x} {- a b}}}
            {with {a {+ (-3) (-3)}}
                {with {b {+ a a}}
                    {foo 1}}}}}}
    \end{lstlisting}
    \begin{itemize}
        \item Evaluación glotona y alcance estático. \textbf{(2 puntos)}\\
        Primero realizamos el ambiente en forma de pila, donde el tope de la pila es el par más reciente que se añadió al ambiente.\\
        Además, de acuerdo a que es evaluación glotona, las expresiones se evalúan tan pronto como sea posible.\\
        Por lo que mostraremos a continuación el ambiente de forma final, es decir, cuando ya es posible evaluar la expresión completa, en este caso, la expresión \texttt{(foo 1)}.\\
        \begin{center}

        \begin{tikzpicture}
            % Dibujar la pila
            \draw (0,0) -- (0,6.6) -- (4.5,6.6) -- (4.5,0); % Contorno de la pila
            \draw (0,0) -- (4.5, 0);
            \draw (0,1.1) -- (4.5,1.1);
            \draw (0,2.2) -- (4.5,2.2);
            \draw (0,3.3) -- (4.5,3.3);
            \draw (0,4.4) -- (4.5,4.4);
            \draw (0,5.5) -- (4.5,5.5);

            % Insertar el texto dentro de la pila (de abajo hacia arriba)
            \node at (2.25,0.55) {a \qquad ( 0 )};
            \node at (2.25,1.65) {b \qquad ( - a a )};
            \node at (2.25,2.75) {x \qquad ( 1 )};
            \node at (2.25,3.85) {foo \quad (fun (x) (- a b))};
            \node at (2.25,4.95) {a \qquad ( -6 )};
            \node at (2.25,6.05) {b \qquad ( + a a )};
            \node at (2.25,6.6) {};
        \end{tikzpicture}
        \end{center}

        Una vez hecho la evaluación glotona y obtenido el ambiente final, procedemos a evaluar la expresión \texttt{(foo 1)}.\\
        Buscamos el valor de \texttt{foo} en el ambiente y obtenemos la función \texttt{(fun (x) (- a b))}.\\
        Asignamos el valor \texttt{1} al parámetro \texttt{x} de la función \texttt{foo}, pero en este caso, dado que es alcance estático, añadimos el par \texttt{(x 1)} justo después de la función \texttt{foo} en el ambiente de la pila, por eso mismo dicha \texttt{x} ya aparece en la pila pues es el ambiente final el que mostramos.\\
        Entonces, evaluamos el cuerpo de la función \texttt{foo} que es \texttt{(- a b)}. Para esto, buscamos el valor de \texttt{a} con alcanze estático en el ambiente y obtenemos que la \texttt{a = 0}.\\
        Luego, buscamos el valor de \texttt{b} en el ambiente y obtenemos que la \texttt{b = - a a} por lo que nuevamente buscamos el valor de \texttt{a} en el ambiente y obtenemos que la \texttt{a = 0}. Entonces, evaluamos la expresión \texttt{- a a} usando el valor de \texttt{a} que acabamos de obtener, y nos queda que, \texttt{b = - a a = - 0 a = - 0 0 = 0}.\\
        Nos queda evaluar la función \texttt{(- a b)} usando los valores que obtuvimos antes: \texttt{(- 0 0) = 0}. Por lo tanto, el resultado final es \textbf{0}.\\

        \item Evaluación glotona y alcance dinámico. \textbf{(2 puntos)}

        Primero realizamos el ambiente en forma de pila, donde el tope de la pila es el par más reciente que se añadió al ambiente.\\
        Además, de acuerdo a que es evaluación glotona, las expresiones se evalúan tan pronto como sea posible.\\
        Por lo que mostraremos a continuación el ambiente de forma final, es decir, cuando ya es posible evaluar la expresión completa, en este caso, la expresión \texttt{(foo 1)}.\\
        \begin{center}

        \begin{tikzpicture}
            % Dibujar la pila
            \draw (0,0) -- (0,6.6) -- (4.5,6.6) -- (4.5,0); % Contorno de la pila
            \draw (0,0) -- (4.5, 0);
            \draw (0,1.1) -- (4.5,1.1);
            \draw (0,2.2) -- (4.5,2.2);
            \draw (0,3.3) -- (4.5,3.3);
            \draw (0,4.4) -- (4.5,4.4);
            \draw (0,5.5) -- (4.5,5.5);

            % Insertar el texto dentro de la pila (de abajo hacia arriba)
            \node at (2.25,0.55) {a \qquad ( 0 )};
            \node at (2.25,1.65) {b \qquad ( - a a )};
            \node at (2.25,2.75) {foo \quad (fun (x) (- a b))};
            \node at (2.25,3.85) {a \qquad ( -6 )};
            \node at (2.25,4.95) {b \qquad ( + a a )};
            \node at (2.25,6.05) {x \qquad ( 1 )};
        \end{tikzpicture}
        \end{center}

        Ya dimos el ambiente final, procedemos a evaluar la expresión \texttt{(foo 1)}.\\
        Buscamos el valor de \texttt{foo} en el ambiente y obtenemos la función \texttt{(fun (x) (- a b))}.\\
        Asignamos el valor \texttt{1} al parámetro \texttt{x} de la función \texttt{foo}, pero en este caso, dado que es alcance dinámico, añadimos el par \texttt{(x 1)} al tope de la pila, la pila que mostramos ya es el ambiente final por lo que dicha \texttt{x} ya aparece en la pila.\\
        Entonces, evaluamos el cuerpo de la función \texttt{foo} que es \texttt{(- a b)}. Para esto, buscamos el valor de \texttt{a} con alcanze dinámico en el ambiente y obtenemos primero que la \texttt{a = -6}.\\
        Luego, buscamos el valor de \texttt{b} en el ambiente y obtenemos que la \texttt{b = + a a} por lo que nuevamente buscamos el valor de \texttt{a} en el ambiente y obtenemos que la \texttt{a = -6}. Entonces, evaluamos la expresión \texttt{+ a a} usando el valor de \texttt{a} que acabamos de obtener, y nos queda que, \texttt{b = + a a = + -6 a = + -6 -6 = -12}.\\
        Nos queda evaluar la función \texttt{(- a b)} usando los valores que obtuvimos antes: \texttt{(- -6 -12) = 6}. Por lo tanto, el resultado final es \textbf{6}.\\


        \item Evaluación perezosa y alcance estático. \textbf{(2 puntos)}

        Primero realizamos el ambiente en forma de pila, donde el tope de la pila es el par más reciente que se añadió al ambiente.\\
        Además, de acuerdo a que es evaluación perezosa, las expresiones se evalúan hasta el final.\\
        Por lo que mostraremos a continuación el ambiente de forma final, es decir, cuando ya es posible evaluar la expresión completa, en este caso, la expresión \texttt{(foo 1)}.\\

        \begin{center}

        \begin{tikzpicture}
            % Dibujar la pila
            \draw (0,0) -- (0,6.6) -- (4.5,6.6) -- (4.5,0); % Contorno de la pila
            \draw (0,0) -- (4.5, 0);
            \draw (0,1.1) -- (4.5,1.1);
            \draw (0,2.2) -- (4.5,2.2);
            \draw (0,3.3) -- (4.5,3.3);
            \draw (0,4.4) -- (4.5,4.4);
            \draw (0,5.5) -- (4.5,5.5);

            % Insertar el texto dentro de la pila (de abajo hacia arriba)
            \node at (2.25,0.55) {a \qquad ( - 5 5 )};
            \node at (2.25,1.65) {b \quad ( - a a )};
            \node at (2.25,2.75) {x \quad (1)};
            \node at (2.25,3.85) {foo \quad (fun (x) ( - a b ) ) };
            \node at (2.25,4.95) {a \qquad ( + (-3) (-3) )};
            \node at (2.25,6.05) {b \qquad (+ a a )};
        \end{tikzpicture}
        \end{center}
        Una vez hecho la evaluación perezosa y obtenido el ambiente final, procedemos a evaluar la expresión \texttt{(foo 1)}.\\
        Buscamos el valor de \texttt{foo} en el ambiente y obtenemos la función \texttt{(fun (x) (- a b))}.\\
        Asignamos el valor \texttt{1} al parámetro \texttt{x} de la función \texttt{foo}, pero en este caso, dado que es alcance estático, añadimos el par \texttt{(x 1)} justo después de la función \texttt{foo} en el ambiente de la pila, por eso mismo dicha \texttt{x} ya aparece en la pila pues es el ambiente final el que mostramos.\\
        Entonces, evaluamos el cuerpo de la función \texttt{foo} que es \texttt{(- a b)}. Para esto, buscamos el valor de \texttt{a} con alcanze estático en el ambiente y obtenemos \texttt{a = (- 5 5)}.\\
        Luego, buscamos el valor de \texttt{b} en el ambiente y obtenemos que la \texttt{b = - a a}, de tal manera que tenemos \texttt{(- (- 5 5) (- a a))} por lo que nuevamente buscamos el valor de \texttt{a} en el ambiente y sustituimos, por lo que obtenemos: \texttt{(- (- 5 5) (- (- 5 5) (- 5 5)))}.\\
        Nos queda evaluar la función \texttt{(- a b)} usando los valores que obtuvimos antes:
        \\\texttt{> (- (- 5 5) (- (- 5 5) (- 5 5)))}
        \\\texttt{> (- (- 5 5) (- (- 5 5) (0)))}
        \\\texttt{> (- (- 5 5) (- 0 0))}
        \\\texttt{> (- (- 5 5) 0)}
        \\\texttt{(- 0 0) = 0}.\\Por lo tanto, el resultado final es \textbf{0}.\\

        \item Evaluación perezosa y alcance dinámico. \textbf{(2 puntos)}

        Primero realizamos el ambiente en forma de pila, donde el tope de la pila es el par más reciente que se añadió al ambiente.\\
        Como es evaluacón perezosa, las expresiones se evalúan hasta que se necesite obtener un único valor..\\
        Por lo que mostraremos a continuación el ambiente de forma final, es decir, cuando ya es posible evaluar la expresión completa, en este caso, la expresión \texttt{(foo 1)}.\\


        \begin{center}

        \begin{tikzpicture}
            % Dibujar la pila
            \draw (0,0) -- (0,6.6) -- (4.5,6.6) -- (4.5,0); % Contorno de la pila
            \draw (0,0) -- (4.5, 0);
            \draw (0,1.1) -- (4.5,1.1);
            \draw (0,2.2) -- (4.5,2.2);
            \draw (0,3.3) -- (4.5,3.3);
            \draw (0,4.4) -- (4.5,4.4);
            \draw (0,5.5) -- (4.5,5.5);

            % Insertar el texto dentro de la pila (de abajo hacia arriba)
            \node at (2.25,0.55) {a \qquad ( - 5 5 )};
            \node at (2.25,1.65) {b \qquad ( - a a )};
            \node at (2.25,2.75) {foo \quad (fun (x) (- a b))};
            \node at (2.25,3.85) {a \qquad ( + (-3) (-3))};
            \node at (2.25,4.95) {b \qquad ( + a a )};
            \node at (2.25,6.05) {x \qquad ( 1 )};
        \end{tikzpicture}
        \end{center}
        Ya dimos el ambiente final, procedemos a evaluar la expresión \texttt{(foo 1)}.\\
        Buscamos el valor de \texttt{foo} en el ambiente y obtenemos la función \texttt{(fun (x) (- a b))}.\\
        Asignamos el valor \texttt{1} al parámetro \texttt{x} de la función \texttt{foo}, pero en este caso, dado que es alcance dinámico, añadimos el par \texttt{(x 1)} al tope de la pila, la pila que mostramos ya es el ambiente final por lo que dicha \texttt{x} ya aparece en la pila.\\
        Procedemos a evaluar el cuerpo: \texttt{( - a b )}.\\
        Desde el tope de la pila buscamos a \texttt{a}, que es \texttt{( a = (-3) (-3) )}. Al sustituir queda: \textbf{(- (+ (-3) (-3)) b)}. Ahora, procedemos a buscar a \texttt{b} desde el tope de la pila, y encontramos a \texttt{b} con el valor \texttt{(+ a a)}. Sustituyendo en el cuerpo de \texttt{foo} tenemos: \texttt{(- (+ (-3) (-3)) (+ a a))}. Al volver a buscar \texttt{a} y volver a sustituir obtenemos: \texttt{(- (+ (-3) (-3)) (+ (+ (-3) (-3)) (+ (-3) (-3))))}.\\
        Simplificando tenemos:\\ \texttt{> (- (+ (-3) (-3)) (- (+ (-3) (-3)) (-6)))}
        \\ \texttt{> (- (+ (-3) (-3)) (+ (-6) (-6)))}\\
        \texttt{> (- (+ (-3) (-3)) (-12))}
        \\ \texttt{> (- (-6) (-12))}
        \\ \texttt{ > (6)}\\
        El resultado final es \texttt{6}.

    \end{itemize}
\end{enumerate}

\end{document}
