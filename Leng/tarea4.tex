\documentclass[11pt]{article}
\usepackage[spanish]{babel}
\usepackage[utf8]{inputenc}
\usepackage{amsmath, amssymb}
\usepackage{geometry}
\usepackage{setspace}
\usepackage{enumitem}
\usepackage{graphicx}
\usepackage{xurl}
\usepackage{listings}
\usepackage{tikz}
\usepackage{xcolor}
\usetikzlibrary{decorations.pathreplacing, calc}

\geometry{letterpaper, margin=2cm}

\begin{document}
\begin{titlepage}
    \centering
    \vspace{2cm}
    {\includegraphics[height=3.2cm]{../logo_unam.png}}
    \hfill
    {\includegraphics[height=3.2cm]{../logo_fc.png}\par}
    \vspace{1cm}
    {\bfseries\LARGE UNIVERSIDAD NACIONAL AUTÓNOMA DE MÉXICO \par}
    \vspace{0.7cm}
    {\scshape\Large FACULTAD DE CIENCIAS \par}
    \vspace{1cm}
    {\itshape\Large Lenguajes de programación \par}
    \vspace{0.5cm}
    {\itshape\Large Semestre 2026-1 \par}
    \vspace{2cm}
    {\scshape\Huge Tarea 4 \par}
    \vspace{1cm}
    {\itshape\Large Fecha de entrega: 20 de octubre de 2025 \par}
    \vspace{2cm}
    {\Large Autores: \par}
    \vspace{0.4cm}
    {\Large Escobar Gonzalez Isaac Giovani \hspace{1cm} 321336400 \par}
    {\Large Garduño Escobar Kevin Jonathan \hspace{0.5cm} 321070629 \par}
    {\Large Zaldivar Alanis Rodrigo \hspace{2.75cm} 424029605 \par}
\end{titlepage}
\section*{Instrucciones}
\noindent Resolver los siguientes ejercicios de forma clara y ordenada de acuerdo a los lineamientos de entrega de tareas disponibles en la página del curso.\\
\section*{Ejercicios}

\begin{enumerate}[leftmargin=0.8cm]
    \item Currifica cada uno de los siguientes términos y coloca los paréntesis necesarios de acuerdo a la asociatividad de la aplicación.
    \begin{enumerate}
        \item $\lambda$abc.abc\\
        Primero hacemos la currificación:
        \[
            \lambda\text{abc.abc} \equiv _{curry} \lambda \text{a.}\lambda \text{b.}\lambda \text{c.abc}
        \]
        Ahora colocamos los paréntesis necesarios:
        \[
            \lambda \text{a.}(\lambda \text{b.}(\lambda \text{c.}((\text{ab})\text{c})))
        \]
        \item $\lambda$abc.$\lambda$cde.acbdce\\
        Primero hacemos la currificación:
        \[
            \lambda\text{abc.}\lambda\text{cde.acbdce} \equiv _{curry} \lambda \text{a.}\lambda \text{b.}\lambda \text{c.}\lambda \text{c.}\lambda \text{d.}\lambda \text{e.acbdce}
        \]
        Ahora colocamos los paréntesis necesarios:
        \[
            \lambda \text{a.}(\lambda \text{b.}(\lambda \text{c.}(\lambda \text{c.}(\lambda \text{d.}(\lambda \text{e.}(((((\text{ac})\text{b})\text{d})\text{c})\text{e}))))))
        \]
        \item $\lambda$a.(($\lambda$ab.b)($\lambda$cd.d))($\lambda$ef.f)\\
        Primero hacemos la currificación:
        \[
            \lambda\text{a.}((\lambda\text{ab.b})(\lambda\text{cd.d}))(\lambda\text{ef.f}) \equiv _{curry} \lambda \text{a.}((\lambda \text{a.}\lambda \text{b.b})(\lambda \text{c.}\lambda \text{d.d}))(\lambda \text{e.}\lambda \text{f.f})
        \]
        Ahora colocamos los paréntesis necesarios:
        \[
            \lambda \text{a.}((\lambda \text{a.}(\lambda \text{b.b}))(\lambda \text{c.}(\lambda \text{d.d})))(\lambda \text{e.}(\lambda \text{f.f}))
        \]
    \end{enumerate}
    \item Aplica $\alpha$-conversiones en cada expresión para obtener una expresión equivalente a la original.
    \begin{enumerate}
        \item $\lambda$a.$\lambda$b.(($\lambda$b.v) ($\lambda$v.b))
        \item $\lambda$a.(a($\lambda$b.(($\lambda$a.a) b)a))
        \item $\lambda$a.(($\lambda$b.a) $\lambda$b.($\lambda$a.a b))
    \end{enumerate}
    \item Aplica $\beta$-reducciones a las siguientes expresiones para llegar a una Forma Normal, en caso de que no tenga forma normal justifica tu respuesta. Además indica en cada paso el \textit{reducto} y el \textit{redex}.\\
    \\
    $l = _{def}\lambda$x.x\\
    $\Omega = _{def}$($\lambda$x.xx)($\lambda$x.xx)
    \begin{enumerate}
        \item $\lambda$x.((xK)$\Omega$)\\
        \textbf{Solución:}\\
        De acuerdo con las reglas de la $\beta$-reducción procedemos a reducir la expresión dada y a dar el redex y reducto en cada paso, empezamos por sustituir $K$ y $\Omega$ en la expresión original:

        \begin{align*}
            & \lambda x.((xK)\Omega)= \lambda x. ((x(\lambda x.\lambda y.x)) (\lambda x.xx) (\lambda x.xx))
        \end{align*}

        Vemos e identificamos que a partir de la sustitución de la definición de $\Omega$ es posible realizar una $\beta$-reducción, pues tenemos una aplicación de la forma $(\lambda x.xx)$, por lo que procedemos a realizar la reducción:

        \begin{align*}
            & \lambda x.((x(\lambda x.\lambda y.x)) (\lambda x.xx) (\lambda x.xx)) \xrightarrow{\beta} \lambda x.((x(\lambda x.\lambda y.x)) (\lambda x.xx)(\lambda x.xx))\\
            & \text{Redex: } ((\lambda x.xx)(\lambda x.xx))\\
            & \text{Reducto: } xx[x:= (\lambda x.xx)]
        \end{align*} 

        Como observamos, nuevamente tenemos una nueva aplicación de la forma $(\lambda x.xx)$, por lo que podemos realizar otra $\beta$-reducción:

        \begin{align*}
            & \lambda x.((x(\lambda x.\lambda y.x)) ((\lambda x.xx)(\lambda x.xx)))\xrightarrow{\beta} \lambda x.((x(\lambda x.\lambda y.x)) (\lambda x.xx)(\lambda x.xx))\\
            & \text{Redex: } ((\lambda x.xx)(\lambda x.xx))\\
            & \text{Reducto: } xx[x:= (\lambda x.xx)] 
        \end{align*}

        Nuevamente, tenemos la misma situación, por lo que podemos observar que este proceso se repite indefinidamente, es decir, siempre tendremos un redex de la forma $((\lambda x.xx)(\lambda x.xx))$ y su respectivo reducto $xx[x:= (\lambda x.xx)]$.\\

        Por lo tanto, la expresión no tiene forma normal pues, como mostramos, la reducción continúa indefinidamente gracias a como está definido $\Omega$: un combinador $\Omega$ nunca llega a una forma normal. 


        \item ($\lambda$x.x(l l))z \\
        \textbf{Solución:}\\
        De acuerdo con las reglas de la $\beta$-reducción procedemos a reducir la expresión dada y a dar el redex y reducto en cada paso, empezamos por sustituir $l$ en la expresión original:
        \begin{align*}
            & (\lambda x.x(l l))z = (\lambda x.x((\lambda x.x)(\lambda x.x)))z
        \end{align*}
        Miramos que es posible realizar una $\beta$-reducción, de dos formas posibles, una es aplicando la función externa y la otra es aplicando la función interna. Procedemos a realizar la reducción aplicando la función externa:
        \begin{align*}
            & (\lambda x.x((\lambda x.x)(\lambda x.x)))z \xrightarrow{\beta} z((\lambda x.x)(\lambda x.x))\\
            & \text{Redex: } (\lambda x.x((\lambda x.x)(\lambda x.x)))z\\
            & \text{Reducto: } x((\lambda x.x)(\lambda x.x))[x:=z]
        \end{align*}
        Ahora, en la nueva expresión obtenida, podemos realizar otra $\beta$-reducción:
        \begin{align*}
            & z((\lambda x.x)(\lambda x.x)) \xrightarrow{\beta} z(\lambda x.x)\\
            & \text{Redex: } (\lambda x.x)(\lambda x.x)\\
            & \text{Reducto: } x[x:= (\lambda x.x)]
        \end{align*}
        Finalmente, en la nueva expresión obtenida, no es posible realizar más $\beta$-reducciones.\\
        Por lo tanto, la forma normal de la expresión es $z(\lambda x.x)$.
        

    \end{enumerate}
\end{enumerate}

\end{document}
