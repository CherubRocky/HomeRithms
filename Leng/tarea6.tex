\documentclass[11pt]{article}
\usepackage[spanish]{babel}
\usepackage[utf8]{inputenc}
\usepackage{amsmath, amssymb}
\usepackage{geometry}
\usepackage{setspace}
\usepackage{enumitem}
\usepackage{graphicx}
\usepackage{xurl}
\usepackage{listings}
\usepackage{tikz}
\usepackage{xcolor}
\usetikzlibrary{decorations.pathreplacing, calc}

\geometry{letterpaper, margin=2cm}

\begin{document}
\begin{titlepage}
    \centering
    \vspace{2cm}
    {\includegraphics[height=3.2cm]{../logo_unam.png}}
    \hfill
    {\includegraphics[height=3.2cm]{../logo_fc.png}\par}
    \vspace{1cm}
    {\bfseries\LARGE UNIVERSIDAD NACIONAL AUTÓNOMA DE MÉXICO \par}
    \vspace{0.7cm}
    {\scshape\Large FACULTAD DE CIENCIAS \par}
    \vspace{1cm}
    {\itshape\Large Lenguajes de programación \par}
    \vspace{0.5cm}
    {\itshape\Large Semestre 2026-1 \par}
    \vspace{2cm}
    {\scshape\Huge Tarea 6 \par}
    \vspace{1cm}
    {\itshape\Large Fecha de entrega: 10 de noviembre de 2025 \par}
    \vspace{2cm}
    {\Large Autores: \par}
    \vspace{0.4cm}
    {\Large Escobar Gonzalez Isaac Giovani \hspace{1cm} 321336400 \par}
    {\Large Garduño Escobar Kevin Jonathan \hspace{0.5cm} 321070629 \par}
    {\Large Zaldivar Alanis Rodrigo \hspace{2.75cm} 424029605 \par}
\end{titlepage}
\section*{Instrucciones}
\noindent Resolver los siguientes ejercicios de forma clara y ordenada de acuerdo a los lineamientos de entrega de tareas disponibles en la página del curso.\\
\section*{Ejercicios}

\begin{enumerate}[leftmargin=0.8cm]
    \item Utilizando Store Passing Style realiza la evaluación de la siguiente expresión y añade su respectivo ambiente y memoria.\\
    \begin{lstlisting}
{with {x 1}
    {with {sub1 {func n {- n 1}}}
        {with {y {newbox 2}}
            {+ {sub1 x} {openbox y}}}}}
    \end{lstlisting}
\end{enumerate}
\textbf{Solución:}\\
Para resolver la expresión dada utilizando Store Passing Style, seguiremos los pasos de evaluación, manteniendo un ambiente y una memoria actualizados en cada paso.\\
Primero daremos el ambiente y la memoria iniciales de acuerdo a la expresión dada.\\
\begin{center}
    \begin{tikzpicture}
        % Dibujar la pila del lado izquierdo (Ambiente Actualizado)
        \node at (2.25, 5.5) {\textbf{Ambiente Inicial}};
        \draw (0,0) -- (0,4.4) -- (4.5,4.4) -- (4.5,0); % Contorno de la pila
        \draw (0,0) -- (4.5, 0);
        \draw (0,1.1) -- (4.5,1.1);
        \draw (0,2.2) -- (4.5,2.2);
        \draw (0,3.3) -- (4.5,3.3);

        % Insertar el texto dentro de la pila (de abajo hacia arriba)
        \node at (2.25,0.55) {x \qquad  0 };
        \node at (2.25,1.65) {sub1 \qquad  1 };
        \node at (2.25,2.75) {y \qquad  2 };
        \node at (2.25,3.85) {n \qquad  4 };
    \end{tikzpicture}
    \hspace{1cm} % Espacio entre las dos pilas
    \begin{tikzpicture}
        % Dibujar la pila del lado derecho (Memoria 2)
        \node at (3, 6.5) {\textbf{Memoria Inicial}};
        \draw (0,0) -- (0,5.5) -- (6,5.5) -- (6,0); % Contorno de la pila
        \draw (0,0) -- (6, 0);
        \draw (0,1.1) -- (6,1.1);
        \draw (0,2.2) -- (6,2.2);
        \draw (0,3.3) -- (6,3.3);
        \draw (0,4.4) -- (6,4.4);

        % Insertar el texto dentro de la pila (de abajo hacia arriba)
        \node at (3,0.55) {0 \qquad 1 };
        \node at (3,1.65) {1 \qquad (closureV n, \{- n 1\}, ([x 0])) };
        \node at (3,2.75) {2 \qquad \fbox{3}};
        \node at (3,3.85) {3 \qquad 2 };
        \node at (3,4.95) {4 \qquad 1 };
    \end{tikzpicture}
\end{center}

Luego, de construir el ambiente y la memoria iniciales, procederemos a fijarnos en el cuerpo del \textbf{with} más interno de la expresión para asi poder evaluar la expresión completa paso a paso.
Vemos que el cuerpo del \textbf{with} más interno es:
\begin{lstlisting}
{+ {sub1 x} {openbox y}}
\end{lstlisting}
Por lo que procedemos a evaluar cada una de las subexpresiones de la suma por separado.\\
Primero evaluamos la subexpresión \textbf{{sub1 x}}, para ello necesitamos buscar dentro del ambiente el valor asociado a \textbf{sub1}, el cual es la dirección 1. Luego, buscamos en la memoria el valor almacenado en la dirección 1, que es closureV el cual contiene información sobre la función, dado que \textbf{sub1} es una función, ahora necesitamos evaluar el argumento de la función, que en este caso es \textbf{x}, por lo que procedemos a dar el siguiente ambiente y memoria pues es necesario agregar el valor de \textbf{x} junto con su valor asociado en la memoria, a partir de haber encontrado el closureV de \textbf{sub1}.\\
\begin{center}
    \begin{tikzpicture}
        % Dibujar la pila del lado izquierdo (Ambiente Actualizado)
        \node at (2.25, 5.5) {\textbf{Ambiente 2}};
        \draw (0,0) -- (0,4.4) -- (4.5,4.4) -- (4.5,0); % Contorno de la pila
        \draw (0,0) -- (4.5, 0);
        \draw (0,1.1) -- (4.5,1.1);
        \draw (0,2.2) -- (4.5,2.2);
        \draw (0,3.3) -- (4.5,3.3);

        % Insertar el texto dentro de la pila (de abajo hacia arriba)
        \node at (2.25,0.55) {x \qquad  0 };
        \node at (2.25,1.65) {sub1 \qquad  1 };
        \node at (2.25,2.75) {y \qquad  2 };
        \node at (2.25,3.85) {n \qquad  4 };

        % Dibujar flecha que apunta a la fila "sub1 1"
        \draw (-0.5,1.65) -- (0,1.65); % Línea que apunta a la fila "sub1 1"
        \draw[fill=black] (0,1.65) -- (-0.1,1.7) -- (-0.1,1.6) -- cycle; % Triángulo para la punta de la flecha
    \end{tikzpicture}
    \hspace{1cm} % Espacio entre las dos pilas
    \begin{tikzpicture}
        % Dibujar la pila del lado derecho (Memoria 2)
        \node at (3, 6.5) {\textbf{Memoria 2}};
        \draw (0,0) -- (0,5.5) -- (6,5.5) -- (6,0); % Contorno de la pila
        \draw (0,0) -- (6, 0);
        \draw (0,1.1) -- (6,1.1);
        \draw (0,2.2) -- (6,2.2);
        \draw (0,3.3) -- (6,3.3);
        \draw (0,4.4) -- (6,4.4);

        % Insertar el texto dentro de la pila (de abajo hacia arriba)
        \node at (3,0.55) {0 \qquad 1 };
        \node at (3,1.65) {1 \qquad (closureV n, \{- n 1\}, ([x 0])) };
        \node at (3,2.75) {2 \qquad \fbox{3}};
        \node at (3,3.85) {3 \qquad 2 };
        \node at (3,4.95) {4 \qquad 1 };

        % Dibujar flecha que apunta a la fila "1 closureV"
        \draw (-0.5,1.65) -- (0,1.65); % Línea que apunta a la fila "1 closureV"
        \draw[fill=black] (0,1.65) -- (-0.1,1.7) -- (-0.1,1.6) -- cycle; % Triángulo para la punta de la flecha
    \end{tikzpicture}
\end{center}
Ahora procedemos a sustituir el valor de \textbf{x} en la función \textbf{sub1}, por lo que tenemos que primero buscar en el ambiente y pila el valor asociado a \textbf{x}, lo cual forma el siguiente ambiente y memoria.\\
\begin{center}
    \begin{tikzpicture}
        % Dibujar la pila del lado izquierdo (Ambiente Actualizado)
        \node at (2.25, 5.5) {\textbf{Ambiente 3}};
        \draw (0,0) -- (0,4.4) -- (4.5,4.4) -- (4.5,0); % Contorno de la pila
        \draw (0,0) -- (4.5, 0);
        \draw (0,1.1) -- (4.5,1.1);
        \draw (0,2.2) -- (4.5,2.2);
        \draw (0,3.3) -- (4.5,3.3);

        % Insertar el texto dentro de la pila (de abajo hacia arriba)
        \node at (2.25,0.55) {x \qquad  0 };
        \node at (2.25,1.65) {sub1 \qquad  1 };
        \node at (2.25,2.75) {y \qquad  2 };
        \node at (2.25,3.85) {n \qquad  4 };

        % Dibujar flecha que apunta a la fila "n 4"
        \draw (-0.5,3.85) -- (0,3.85); % Línea que apunta a la fila "n 4"
        \draw[fill=black] (0,3.85) -- (-0.1,3.9) -- (-0.1,3.8) -- cycle; % Triángulo para la punta de la flecha
    \end{tikzpicture}
    \hspace{1cm} % Espacio entre las dos pilas
    \begin{tikzpicture}
        % Dibujar la pila del lado derecho (Memoria 3)
        \node at (3, 6.5) {\textbf{Memoria 3}};
        \draw (0,0) -- (0,5.5) -- (6,5.5) -- (6,0); % Contorno de la pila
        \draw (0,0) -- (6, 0);
        \draw (0,1.1) -- (6,1.1);
        \draw (0,2.2) -- (6,2.2);
        \draw (0,3.3) -- (6,3.3);
        \draw (0,4.4) -- (6,4.4);

        % Insertar el texto dentro de la pila (de abajo hacia arriba)
        \node at (3,0.55) {0 \qquad 1 };
        \node at (3,1.65) {1 \qquad (closureV n, \{- n 1\}, ([x 0])) };
        \node at (3,2.75) {2 \qquad \fbox{3}};
        \node at (3,3.85) {3 \qquad 2 };
        \node at (3,4.95) {4 \qquad 1 };

        % Dibujar flecha que apunta a la fila "4 1"
        \draw (-0.5,4.95) -- (0,4.95); % Línea que apunta a la fila "4 1"
        \draw[fill=black] (0,4.95) -- (-0.1,5) -- (-0.1,4.9) -- cycle; % Triángulo para la punta de la flecha
    \end{tikzpicture}
\end{center}
Como observamos las flechas en ambas pilas, el valor asociado a \textbf{x} es 1, por lo que procedemos a sustituir este valor en la función \textbf{sub1}, obteniendo así:
\begin{lstlisting}
{sub1 x} = {- x 1} = {- 1 1} = 0
\end{lstlisting}

Ahora que ya tenemos la parte izquierda de la suma principal, podemos proceder a evaluar la parte derecha de la suma, que es \textbf{{openbox y}}. Para ello, buscamos en el ambiente el valor asociado a \textbf{y}, lo cual nos genera el siguiente ambiente y memoria.\\
\begin{center}
    \begin{tikzpicture}
        % Dibujar la pila del lado izquierdo (Ambiente Actualizado)
        \node at (2.25, 5.5) {\textbf{Ambiente 4}};
        \draw (0,0) -- (0,4.4) -- (4.5,4.4) -- (4.5,0); % Contorno de la pila
        \draw (0,0) -- (4.5, 0);
        \draw (0,1.1) -- (4.5,1.1);
        \draw (0,2.2) -- (4.5,2.2);
        \draw (0,3.3) -- (4.5,3.3);

        % Insertar el texto dentro de la pila (de abajo hacia arriba)
        \node at (2.25,0.55) {x \qquad  0 };
        \node at (2.25,1.65) {sub1 \qquad  1 };
        \node at (2.25,2.75) {y \qquad  2 };
        \node at (2.25,3.85) {n \qquad  4 };

        % Dibujar flecha que apunta a la fila "y 2"
        \draw (-0.5,2.75) -- (0,2.75); % Línea que apunta a la fila "y 2"
        \draw[fill=black] (0,2.75) -- (-0.1,2.8) -- (-0.1,2.7) -- cycle; % Triángulo para la punta de la flecha
    \end{tikzpicture}
    \hspace{1cm} % Espacio entre las dos pilas
    \begin{tikzpicture}
        % Dibujar la pila del lado derecho (Memoria 4)
        \node at (3, 6.5) {\textbf{Memoria 4}};
        \draw (0,0) -- (0,5.5) -- (6,5.5) -- (6,0); % Contorno de la pila
        \draw (0,0) -- (6, 0);
        \draw (0,1.1) -- (6,1.1);
        \draw (0,2.2) -- (6,2.2);
        \draw (0,3.3) -- (6,3.3);
        \draw (0,4.4) -- (6,4.4);

        % Insertar el texto dentro de la pila (de abajo hacia arriba)
        \node at (3,0.55) {0 \qquad 1 };
        \node at (3,1.65) {1 \qquad (closureV n, \{- n 1\}, ([x 0])) };
        \node at (3,2.75) {2 \qquad \fbox{3}};
        \node at (3,3.85) {3 \qquad 2 };
        \node at (3,4.95) {4 \qquad 1 };

        % Dibujar flecha que apunta a la fila "2 3"
        \draw (-0.5,2.75) -- (0,2.75); % Línea que apunta a la fila "2 3"
        \draw[fill=black] (0,2.75) -- (-0.1,2.8) -- (-0.1,2.7) -- cycle; % Triángulo para la punta de la flecha

        % Dibujar flecha que apunta a la fila "3 2"
        \draw (-0.5,3.85) -- (0,3.85); % Línea que apunta a la fila "3 2"
        \draw[fill=black] (0,3.85) -- (-0.1,3.9) -- (-0.1,3.8) -- cycle; % Triángulo para la punta de la flecha
    \end{tikzpicture}
\end{center}
Como se observa, el valor asociado a \textbf{y} es la dirección 2, por lo que buscamos en la memoria el valor almacenado en la dirección 2, el cual es una caja que contiene la referencia de memoria 3, por lo que buscamos en la memoria el valor almacenado en la dirección 3, el cual es 2. Por lo tanto, \textbf{{openbox y}} evalúa a 2.\\
Una vez que tenemos dicho valor, procedemos a terminar de evaluar la suma completa:
\begin{lstlisting}
    {+ 0 { openbox y}} = {+ 0 2} = 2
\end{lstlisting}
Por lo tanto, la evaluación completa de la expresión original es \textbf{2} con sus respectivos ambientes y memorias en cada paso de la evaluación junto con sus respectivas flechas indicadoras.\\


\end{document}